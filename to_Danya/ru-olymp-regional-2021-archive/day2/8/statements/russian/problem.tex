\begin{problem}{A+B}{стандартный ввод}{стандартный вывод}{1 секунда}{512 мегабайт}

Рассмотрим $a$, $b$ и $c$~--- целые неотрицательные числа, записанные в десятичной системе счисления. Пусть они имеют одинаковую длину $n$, при этом запись может начинаться с нуля. Числа записаны одно под другим, цифры расположены в три строки и $n$ столбцов. Рассмотрим пример такой записи:

\begin{center}
\texttt{01211}\\
\texttt{12099}\\
\texttt{23300}
\end{center}

Требуется переставить столбцы в этой записи таким образом, чтобы выполнялось равенство $a+b=c$. В полученной записи ведущие нули уже запрещены. Сколько существует различных способов это сделать?

Перестановки столбцов считаются различными, даже если полученные записи совпадают. Например, если в записи выше переставить два последних столбца, получится другая перестановка, хотя цифры в этих колонках совпадают.

Поскольку ответ может быть довольно большим, требуется посчитать для него остаток по модулю $10^9+7$.

\InputFile
Во входных данных записаны целые неотрицательные числа $a$, $b$ и $c$ по одному в строке. Каждое число состоит из $n$ десятичных цифр и может начинаться с нуля ($2 \leq n \leq 2 \cdot 10^5$).

\OutputFile
Выведите количество подходящих перестановок столбцов по модулю $10^9+7$.


\Scoring
Баллы за каждую подзадачу начисляются только в случае, если все тесты для этой
подзадачи и необходимых подзадач успешно пройдены.

\begin{center}
\renewcommand{\arraystretch}{1.3}
\begin{tabular}{|c|c|c|c|c|}
\hline
\textbf{Подзадача} & 
\textbf{Баллы} & 
\textbf{Ограничения} & 
\parbox{3cm}{\textbf{\centering\\Необходимые\\подзадачи\\\vspace{2mm}}} & 
\parbox{3cm}{\textbf{\centering\\Информация\\о проверке\\\vspace{2mm}}} 
\\ \hline
1 & 7 & $2 \leq n \leq 6$ & & первая ошибка \\
\hline
2 & 14 & $2 \leq n \leq 18$ & 1  & первая ошибка\\
\hline
3 & 15 & $2 \leq n \leq 200$, нет цифры ноль &  & первая ошибка\\
\hline
4 & 5 & $2 \leq n \leq 200$ & 1--3  & первая ошибка\\
\hline
5 & 17 & $2 \leq n \leq 750$, нет цифры ноль & 3  & первая ошибка\\
\hline
6 & 5 & $2 \leq n \leq 750$ & 1--5  & первая ошибка\\
\hline
7 & 20 & $2 \leq n \leq 2 \cdot 10^5$, нет цифры ноль & 3, 5  & первая ошибка\\
\hline
8 & 17 & $2 \leq n \leq 2 \cdot 10^5$ & 1--7 & первая ошибка\\
\hline
\end{tabular}
\end{center}

\Examples

\begin{example}
\exmpfile{example.01}{example.01.a}%
\exmpfile{example.02}{example.02.a}%
\exmpfile{example.03}{example.03.a}%
\exmpfile{example.04}{example.04.a}%
\end{example}

\Explanations
В первом примере подходят все перестановки столбцов.

Во втором примере единственная подходящая перестановка~--- $10+20=30$. $01+02=03$ не считается из-за наличия ведущих нулей.

В третьем примере возможны варианты $10121+21909=32030$ и $12101+20919=33020$, причём каждый из них может быть получен двумя разными перестановками.

\end{problem}

