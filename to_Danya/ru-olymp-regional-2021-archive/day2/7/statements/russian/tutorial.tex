\begin{tutorial}{Хорошие раскраски}

\medskip
\textit{Автор задачи: Ильдар Гайнуллин}
\medskip

Эта задача "--- довольно нетрадиционная для олимпиад по информатике. Вместо конкретного конструктивного или алгоритмического решения участникам предлагалось поэкспериментировать с эвристиками и перебором.

Прежде чем разобрать основные идеи, отметим, что конструктивного решения в этой задаче, скорее всего, нет. На это наводит следующая мысль: конструктивный паттерн, который решал бы задачу для приведенных ограничений, мог бы быть распространен на бесконечное поле. А можно доказать "--- кстати, интересное олимпиадное упражнение по математике "--- что раскрасить таким образом в 2 или 3 цвета бесконечное поле невозможно.

Отметим, что в задаче очень маленькие ограничения и потестовая оценка. В принципе, можно локально запустить решение для всех возможных входов (их 200) и отправить в систему результаты предподсчета. Практически любое продвижение для больших полей вознаграждается баллами.

Перейдем теперь к рассмотрению основных идей переборных решений и решений с помощью локальных оптимизаций.

\subsection*{Перебор}
Самое простое решение это перебор за $\mathcal{O}(c^{n \cdot m})$, можно перебирать все раскраски простой рекурсией: перебирать цвета клеток в порядке сортировки по строкам и при равенстве по столбцам, и проверять, подходит ли итоговое поле. Чтобы оптимизировать это решение, в процессе перебора подходящих раскрасок, можно проверять, что среди клеток, которым уже проставлены значения, нет плохих четверок. Это решение уже может пройти все тесты с $c=2$ и значительное число тестов с $c=3$. Чтобы еще больше оптимизировать решение, можно перебирать цвета для клеток в случайном порядке. 

Дополнительная оптимизация, которая сильно ускоряет перебор, "--- добавление <<\textit{мемоизации}>>. А именно, давайте запоминать некоторую информацию про текущую раскраску, которая точно не приведет к ответу (так как такая же комбинация перебиралась ранее, и ответ найден не был). Например, можно хранить множество троек $(c, y_1, y_2)$, что среди уже покрашенных клеток найдется $x$, что $a_{r,y_1}=a_{r,y_2}=c$, и проверять, что такое множество не встречалось, когда мы начинаем красить первую клетку в определенной строке.

Прошлое решение работает заметно лучше при $m<n$, в противном случае можно поменять местами $n$ и $m$ и транспонировать поле для ответа. С этими оптимизациями можно пройти почти все тесты.

\subsection*{Локальные оптимизации}
Другим возможным решением является метод локальных оптимизаций.

Исходно проставим каждой клетке случайный цвет $1 \ldots c$. Затем будем выбирать случайную клетку и изменять ей цвет, если при этом уменьшиться \textit{стоимость} поля.

Стоимостью поля может быть почти любая функция, которая будет равна нулю для хороших полей. Например, за нее можно взять количество четверок $x_1, x_2, y_1, y_2$, что $1 \leq x_1 < x_2 \leq n$, $1 \leq y_1 < y_2 \leq m$, и клетки $(x_1, y_1)$, $(x_2, y_1)$, $(x_1, y_2)$ и $(x_2, y_2)$ покрашены в одинаковый цвет.
Когда никакое изменение цвета не приведет к уменьшению ответа, можно завершить процесс.

Для большинства случайных раскрасок стоимость в конечном итоге будет равна маленькому числу. А если с исходной раскраской <<\textit{повезёт}>>, то стоимость будет равна нулю, и поле сойдется к хорошему полю.
Отсюда выходит следующее решение: будем случайно генерировать исходные поля, запускать описанный ваше процесс, пока стоимость уменьшается, и проверять что в конце получился $0$. Если получился, то можно вывести ответ. Иначе, генерируем исходное поле заново.

Это решение способно пройти все тесты с ограничениями этой задачи меньше чем за секунду.


\end{tutorial}
