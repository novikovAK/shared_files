\begin{problem}{Хорошие раскраски}{стандартный ввод}{стандартный вывод}{1 секунда}{512 мегабайт}

Назовем раскраску клеток таблицы $n \times m$ \textit{хорошей}, если никакие четыре клетки, центры которых образуют вершины прямоугольника со сторонами, параллельными осям координат, не покрашены в один цвет.

Иначе говоря, для раскраски не должно быть четверки целых чисел $x_1, x_2, y_1, y_2$, что $1 \leq x_1 < x_2 \leq n$, $1 \leq y_1 < y_2 \leq m$, и клетки $(x_1, y_1)$, $(x_2, y_1)$, $(x_1, y_2)$ и $(x_2, y_2)$ покрашены в одинаковый цвет.

Требуется написать программу, которая по заданным целым числам $n$, $m$ и $c$ находит любую хорошую раскраску таблицы $n \times m$ в $c$ цветов. 

\InputFile
В первой строке записаны три целых числа $n, m, c$ ($2 \leq n, m \leq 10$, $2 \leq c \leq 3$).

Гарантируется, что для заданных во входных данных значений существует хотя бы одна хорошая раскраска.

\OutputFile
Выведите $n$ строк по $m$ чисел в каждой. 

В качестве $j$-го числа $i$-й строки выведите $a_{i,j}$ "--- цвет клетки $(i,j)$ ($1 \leq a_{i,j} \leq c$).

Если есть несколько хороших раскрасок, можно вывести любую из них.

\Scoring
Кроме теста из примера в этой задаче $20$ тестов, каждый независимо оценивается в $5$ баллов. Среди этих тестов в пяти тестах $c = 2$ и в пятнадцати тестах $c = 3$.

Для каждого теста сообщается результат проверки на этом тесте.

\Example

\begin{example}
\exmpfile{example.01}{example.01.a}%
\end{example}

\end{problem}

