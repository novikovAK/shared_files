\begin{tutorial}{Числа}

\medskip
\textit{Автор задачи: Андрей Станкевич}
\medskip

\subsection*{Подзадача 1}
Заметим, что $k = 0$ означает, что число должно состоять из всех одинаковых цифр. В этой подзадаче можно просто увеличивать $x$, пока все цифры числа не будут одинаковы. Число $111\,111$ обладает этим свойством и не меньше всех возможных чисел во вводе, значит будет сделано не больше чем такое количество шагов.

\subsection*{Подзадача 2}
В этой подзадаче увеличение шагами по одному уже не приводит к успеху. Для получения баллов за эту подзадачу нужно разобраться, как устроено минимальное число, большее заданного, состоящее из одинаковых цифр. 

Первую цифру числа нельзя уменьшить, посмотрим, можно ли её оставить такой же.
Пойдем по числу от старших цифр к младшим, пока они равны первой цифре числа $f$. Если в какой-то момент очередная цифра цифра отличается от первой, посмотрим меньше она или больше. Если она оказывается меньше первой цифры числа, то можно просто заменить все цифры на $f$, получив большее число из всех одинаковых цифр. Если же она оказывается больше, то придется увеличить первую цифру. Это можно сделать, если она не равна 9. В этом случае заполним число цифрами $f+1$. Иначе надо взять число из всех единиц, имеющее на одну цифру в своей десятичной записи больше.

\subsection*{Подзадача 3}
В этой задаче снова маленькое значение $x$, и можно увеличивать его, пока все его цифры, кроме, может быть, одной не станут равны. На этот раз верхним порогом выступает число $100\,000$. Единственная трудность этой подзадачи --- техническая, проверить, что все цифры, кроме не более чем $k$, равны.

\subsection*{Подзадача 4}
На самом деле полное решение задачи даже проще разбора частных случаев предыдущих подзадач. Заметим, что искомое число имеет не более 18 десятичных цифр. Значит всего подходящих чисел $18\cdot 9$ для $k = 0$ и не больше $18\cdot 10 \cdot 18 \cdot 10$ для $k=1$ (во втором случае мы перебираем длину, основную цифру, позицию отличающейся и саму отличающуюся). Переберем все потенциальные ответы и из тех, которые не меньше $x$, выберем минимальный. 

Код на C++, который строит число из цифр $d$ длины $len$, на позиции $pos$ ставится цифра $d2$:

\begin{lstlisting}[language=C++]
long long num(int len, int d, int pos, int d2) {
    long long res = 0;
    for (int i = 0; i < len; i++) {
        if (i == pos) {
            res = res * 10 + d2;
        } else {
            res = res * 10 + d;
        }
    }
    return res;
}
\end{lstlisting}

Код на C++, который перебирает все подходящие числа для $k=1$:

\begin{lstlisting}[language=C++]
for (int i = 1; i <= 18; i++) {
    for (int d = 0; d < 10; d++) {
        for (int p = 0; p < i; p++) {
            for (int d2 = 0; d2 < 10; d2++) {
                long long y = num(i, d, p, d2);
                if (y >= x && (ans == -1 || y < ans)) {
                    ans = y;
                }
            }
        }
    }
}

\end{lstlisting}

\end{tutorial}
