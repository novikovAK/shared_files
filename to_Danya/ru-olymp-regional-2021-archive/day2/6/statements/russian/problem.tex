\begin{problem}{Числа}{стандартный ввод}{стандартный вывод}{1 секунда}{512 мегабайт}

Аня любит, когда числа состоят из одинаковых цифр. Поэтому ей нравятся числа 777 или 5555, а вот число 1234 ей совсем не нравится. 

Иногда у Ани бывает хорошее настроение, тогда ей по прежнему нравятся все числа, состоящие из одинаковых цифр, но также нравятся числа, в которых все цифры кроме одной одинаковые, как, например, в числе 77727. 

У Ани есть число $x$. Аня хочет найти минимальное целое число $y \ge x$, которое ей понравится.

Требуется написать программу, которая по заданному целому числу $x$ и информации, хорошее ли настроение у Ани, находит минимальное целое число $y \ge x$, которое нравится Ане. 

\InputFile
Первая строка ввода содержит целое число $x$ ($1 \le x \le 10^{17}$, обратите внимание, что число $x$ не может быть сохранено в стандартном 32-битном типе данных, необходимо использовать 64-битный тип данных, например <<\texttt{long long}>> в C++, <<\texttt{int64}>> в Паскале).

Вторая строка ввода содержит число $k$, равное $0$ или $1$. Значение $k = 1$ означает, что у Ани хорошее настроение, а значение $k = 0$ "--- что это не так.

\OutputFile
Следует вывести одно целое число $y$.

Должны выполняться следующие свойства: 
\begin{itemize}[noitemsep,leftmargin=2cm]\vspace{-0.3cm}
\item $y \ge x$;
\item если $k = 0$, то все цифры в десятичной записи числа $y$ должны совпадать;
\item если $k = 1$, то все цифры в десятичной записи числа $y$, кроме, может быть, одной, должны совпадать.
\end{itemize}


\Scoring
Баллы за каждую подзадачу начисляются только в случае, если все тесты для этой
подзадачи и необходимых подзадач успешно пройдены.

\begin{center}
\renewcommand{\arraystretch}{1.3}
\begin{tabular}{|c|c|c|c|c|}
\hline
\textbf{Подзадача} & 
\textbf{Баллы} & 
\textbf{Ограничения} & 
\parbox{3cm}{\textbf{\centering\\Необходимые\\подзадачи\\\vspace{2mm}}} & 
\parbox{3cm}{\textbf{\centering\\Информация\\о проверке\\\vspace{2mm}}} 
\\ \hline
1 & 15 & $1 \le x \le 10^5$, $k = 0$ &  & полная \\ \hline
2 & 20 & $1 \le x \le 10^{17}$, $k = 0$ & 1 & первая ошибка \\ \hline
3 & 21 & $1 \le x \le 10^5$, $k = 0$ или $k = 1$ & 1 & полная \\ \hline
4 & 44 & $1 \le x \le 10^{17}$, $k = 0$ или $k = 1$ & 1--3 & первая ошибка \\ \hline
\end{tabular}
\end{center}


\Examples

\begin{example}
\exmpfile{example.01}{example.01.a}%
\exmpfile{example.02}{example.02.a}%
\end{example}

\end{problem}

