\begin{tutorial}{Календарь на Альфе Центавра}

\medskip
\textit{Автор задачи: Андрей Станкевич}
\medskip

Заметим ключевую идею. Пусть мы нашли $x$ --- номер дня от начала летосчисления до заданного во вводе дня. С первого дня первого месяца первого года (который, как мы знаем, имел обозначение <<\t{a}>>, прошло $x - 1$ дней. Каждые $w$ дней повторялся день с обозначением <<\t{a}>>, поэтому на самом деле нас интересует величина $y = (x-1) \bmod w$.  

Заметим, что если $y = 0$, то день имеет обозначение <<\t{a}>>, если $y = 1$, то <<\t{b}>>, и так далее. Чтобы получить по сдвигу относительно <<\t{a}>> букву в английском алфавите, можно воспользоваться возможностями языка программирования. В С++ это сделать проще всего:

\begin{lstlisting}[language=c++]
    cout << 'a' + y << endl;
\end{lstlisting}

В других языках могут потребоваться функции приведения типа. Например, в Паскале:

\begin{lstlisting}[language=Pascal]
    writeln(chr(ord('a') + y));
\end{lstlisting}

В языке Python:

\begin{lstlisting}[language=Python]
    print(chr(ord('a') + y))
\end{lstlisting}

Осталось разобраться, как найти $x$.

\subsection*{Подзадача 1}
В первой подзадаче год состоит из одного дня. Поэтому $i = 1$ и $j = 1$, а значит $x = k$.

\subsection*{Подзадача 2}
Во второй подзадаче $m=1$, то есть месяц всего один. $x = (k-1) \cdot d + i$.
Заметим, что в этой подзадаче благодаря тому, что $k$ небольшое, достаточно 32-битного типа данных.

\subsection*{Подзадача 3}
В третьей подзадаче речь все время идет о первом дне первого месяца. $x = (k-1) \cdot d \cdot m + 1$.

\subsection*{Подзадача 4}
В четвертой подзадаче $k = 1$. Год учитывать не надо, надо учесть только день и месяц. $x = (j - 1) \cdot d + i$.

\subsection*{Подзадачи 5 и 6}
Наконец получим общую формулу: $x = (k - 1) \cdot m \cdot d + (j - 1) \cdot d + i$. Отличие пятой подзадачи в том, что в ней $k$ невелико и поэтому можно использовать 32-битный тип данных.

В заключение отметим, что различные способы итерациями перебирать дни могли проходить некоторые подзадачи, например подзадачи 1, 2, 3, 5.



\end{tutorial}
