\begin{problem}{Календарь на Альфе Центавра}{стандартный ввод}{стандартный вывод}{1 секунда}{512 мегабайт}

На планете в системе Альфы Центавра год состоит из $m$ месяцев, пронумерованных от $1$ до $m$, а каждый месяц из $d$ дней, пронумерованных от $1$ до $d$. В свою очередь неделя у поселенцев на этой планете состоит из $w$ дней, проиндексированных строчными английскими буквами, от <<\t{a}>> до $w$-й буквы английского алфавита. 

Первый день первого месяца первого года соответствует букве <<\t{a}>>.

Требуется определить, какой букве будет соответствовать $i$-й день $j$-го месяца $k$-го года.

\InputFile
Первая строка ввода содержит три целых числа $d$, $m$ и $w$ ($1 \le d, m \le 100$, $1 \le w \le 26$). 

Вторая строка ввода содержит три целых числа $i$, $j$ и $k$ ($1 \le i \le d$, $1 \le j \le m$, $1 \le k \le 10^9$).

\OutputFile
Выведите одну строчную букву английского алфавита "--- какой букве соответствует $i$-й день $j$-го месяца $k$-го года.

\Scoring
Баллы за каждую подзадачу начисляются только в случае, если все тесты для этой
подзадачи и необходимых подзадач успешно пройдены.

\begin{center}
\renewcommand{\arraystretch}{1.3}
\begin{tabular}{|c|c|c|c|c|}
\hline
\textbf{Подзадача} & 
\textbf{Баллы} & 
\textbf{Доп. ограничения} & 
\parbox{3cm}{\textbf{\centering\\Необходимые\\подзадачи\\\vspace{2mm}}} & 
\parbox{3cm}{\textbf{\centering\\Информация\\о проверке\\\vspace{2mm}}} 
\\ \hline
1 & 16 & $d = 1, m = 1$ & & первая ошибка\\
\hline
2 & 16 & $m = 1$, $k \le 10^7$ & 1 & первая ошибка\\
\hline
3 & 17 & $i = 1$, $j = 1$ & & первая ошибка\\
\hline
4 & 17 & $k = 1$ & & первая ошибка\\
\hline
5 & 17 & $k \le 100$ & 4 & первая ошибка\\
\hline
6 & 17 & нет & 1--5& первая ошибка\\
\hline
\end{tabular}
\end{center}

\Example

\begin{example}
\exmpfile{example.01}{example.01.a}%
\end{example}

\Note
Обратите внимание, при решении этой задачи рекомендуется использовать 64-битные типы данных, например <<\t{long long}>> в C++, <<\t{int64}>> в Паскале.

\end{problem}

