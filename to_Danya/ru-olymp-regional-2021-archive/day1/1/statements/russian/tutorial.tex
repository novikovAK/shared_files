\begin{tutorial}{Два станка}

\medskip
\textit{Автор задачи: Даниил Орешников}
\medskip

В первых четырех подзадачах достаточно легко найти конкретную формулу, которая позволяет найти ответ. Обозначим искомое количество деталей как $r$.

\subsection*{Подзадача 1}
Если $a = 0$ и $x = 0$, то первый станок вообще нет смысла использовать. Введем в строй второй и получим ответ: $$r = b \cdot \max(k - b, 0).$$

Важно не забыть, что время на запуск может оказаться больше $k$, и тогда количество деталей будет равно $0$, для чего мы и берем $\max$ в формуле.

\subsection*{Подзадача 2}
Если $a$ и $b$ равны нулю, то можно сразу запустить оба станка в работу и получить ответ: 
$$r = (x + y) \cdot k.$$

\subsection*{Подзадача 3}
Когда $a = b$, выгодно сначала ввести в строй наиболее производительный станок, и сразу после этого начать вводить в строй второй. Ответ тогда будет равен: 
$$r = \max(x, y) \cdot \max(k - a, 0) + \min(x, y) \cdot \max(k - 2a, 0).$$ 

Тут снова надо не забыть о случаях, когда время на введение в строй больше доступного нам.

\subsection*{Подзадача 4}

Если станки производят детали с одинаковой скоростью, то требуется ввести в строй сначала тот, на запуск которого уйдет меньше времени, чтобы детали раньше начали производиться. Получаем: 
$$r = a \cdot \max(k - \min(a, b), 0) + a \cdot \max(k - a - b, 0).$$

\subsection*{Подзадача 5}

Теперь посмотрим как решать общий случай. Переберем два варианта: какой станок будем вводить в строй первым. Если мы сначала введем в строй первый станок, ответ будет: 
$$r_1 = x \cdot \max(k - a, 0) + y \cdot \max(k - a - b, 0).$$ 

Наоборот, если сначала ввести в строй второй, то получим: 
$$r_2 = y \cdot \max(k - b, 0) + x \cdot \max(k - a - b, 0).$$ 

Чтобы получить ответ на задачу, достаточно взять максимум из двух этих значений:
$$r = \max(r_1, r_2).$$

Заметим, что, разумеется, решение пятой подзадачи решает также и первые четыре подзадачи.

\end{tutorial}
