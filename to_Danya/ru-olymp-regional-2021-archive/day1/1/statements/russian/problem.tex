\begin{problem}{Два станка}{стандартный ввод}{стандартный вывод}{1 секунда}{512 мегабайт}

На производстве имеется два станка. Необходимо изотовить как можно больше деталей за сегодняшнюю смену, продложительность которой $k$ минут.

Станки находятся в законсервированном состоянии. Для того, чтобы ввести в строй первый станок, требуется $a$ минут, после чего он будет производить $x$ деталей в минуту. Для того, чтобы ввести в строй второй станок, требуется $b$ минут, после чего он будет производить $y$ деталей в минуту.

Для введения в строй станка требуется присутствие инженера, поэтому нельзя вводить в строй два станка одновременно. При этом введение станка в строй и изготовление деталей на другом станке, а также одновременное изготовление деталей на двух станках разрешается.

Требуется выяснить, какое максимальное количество деталей удастся изготовить за $k$ минут.

\InputFile
В первой строке ввода дано единственное целое неотрицательное число $k$~--- количество минут в смене ($0 \le k \le 10^9$).

Во второй строке ввода даны целые неотрицательные числа $a$ и $x$~--- время введения первого станка в строй и количество деталей, которое он изготавливает за одну минуту ($0 \le a, x \le 10^9$).

В третьей строке ввода даны целые неотрицательные числа $b$ и $y$~--- время введения второго станка в строй и количество деталей, которое он изготавливает за одну минуту ($0 \le b, y \le 10^9$).


\OutputFile
Выведите единственное число~--- максимальное количество деталей, которое удастся изготовить за смену.

Обратите внимание, что ответ в этой задаче может быть довольно большим и не помещаться в 32-битные типы данных. Рекомендуется использовать 64-битный тип данных, например <<\t{\mbox{long long}}>> в C++ или <<\t{int64}>> в Паскале.

\Scoring
Баллы за каждую подзадачу начисляются только в случае, если все тесты для этой
подзадачи и необходимых подзадач успешно пройдены.

\begin{center}
\renewcommand{\arraystretch}{1.3}
\begin{tabular}{|c|c|c|c|c|}
\hline
\textbf{Подзадача} & 
\textbf{Баллы} & 
\textbf{Доп. ограничения} & 
\parbox{3cm}{\textbf{\centering\\Необходимые\\подзадачи\\\vspace{2mm}}} & 
\parbox{3cm}{\textbf{\centering\\Информация\\о проверке\\\vspace{2mm}}} 
\\ \hline
1 & 17 & $a = 0$, $x = 0$ &  & полная \\ \hline
2 & 14 & $a = 0$, $b = 0$ &  & полная \\ \hline
3 & 20 & $a = b$ & 2 & первая ошибка \\ \hline
4 & 20 & $x = y$ & & первая ошибка \\ \hline
5 & 29 & нет & 1 -- 4 & первая ошибка\\ \hline
\end{tabular}
\end{center}


\Example

\begin{example}
\exmpfile{example.01}{example.01.a}%
\end{example}

\Explanation
В примере выгодно сначала ввести в строй второй станок и за оставшиеся $15$ минут изготовить $45$ деталей, а затем ввести в строй первый и за оставшиеся $5$ минут изготовить на нём еще $20$ деталей. 

Если сначала ввести в строй первый станок и изготовить на нем в оставшиеся $10$ минут $40$ деталей, то после введения в строй второго на нем удастся изготовить лишь $15$ деталей, суммарно $55$, что меньше, чем $65$.

\end{problem}

