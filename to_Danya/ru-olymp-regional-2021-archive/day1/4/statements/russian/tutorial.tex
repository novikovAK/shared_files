\begin{tutorial}{Антенна}

\medskip
\textit{Авторы задачи: Даниил Орешников, Геннадий Короткевич}
\medskip

\subsection*{Подзадача 1}
Для того, чтобы решить первую подзадачу, можно перебрать, в каком порядке нужно соединить фрагменты. Всего существует $n!$ таких вариантов. Для каждого варианта можно вычислить позиции всех перекладин и проверить, подходит ли такой вариант.

\subsection*{Подзадача 2}
Для того, чтобы решить вторую подзадачу, нужно заметить, что если на одном из фрагментов присутствует более одной перекладины, то, если существует способ правильно собрать антенну, расстояние между соседними перекладинами должно быть равно расстоянию между перекладинами на этом фрагменте. Поэтому можно заранее проверить, что расстояния между соседними перекладинами в каждом отдельно взятом фрагменте равны. После чего можно оставить в каждом фрагменте только первую и последнюю перекладины. Затем, как в предыдущей подзадаче, перебрать порядок фрагментов и проверить, подходит ли он.

\subsection*{Общая идея}
Как во второй подзадаче, оставим на каждом фрагменте только первую и последнюю перекладины. Пусть на $i$-м фрагменте первая перекладина находится на расстоянии $l_i$ от начала, а последняя~--- на расстоянии $r_i$ от конца. Научимся проверять, существует ли порядок фрагментов, при котором расстояние между соседними перекладинами будет равно $d$. Пусть такой порядок существует, обозначим его $q_1$, $q_2$, \dots $q_n$. Тогда должно выполняться $r_{q_i} + l_{q_{i + 1}} = d$ ($1 \le i \le n - 1$), значит $l_{q_{i + 1}} = d - r_{q_i}$. Рассмотрим граф, вершинами которого являются все целые числа. Проведем ориентированное ребро из $l_{i}$ в $d - r_{i}$ для всех $i$. Заметим, что искомый порядок существует тогда и только тогда, когда в построенном графе существует Эйлеров путь. Поиск Эйлерова пути (и проверка существования) является стандартным алгоритмом.

Например, в первом тесте из примеров, $l_1 = 3$, $r_1 = 4$, $l_2 = 6$, $r_2 = 2$, $l_3 = 1$, $r_3 = 2$. В ответе на тест, $d = 5$. Построим граф по этим данным:

\begin{center}
\includegraphics[width=0.5\textwidth]{pic.0}
\end{center}

Рядом с вершинами написаны соответствующие им числа, а рядом с ребрами~--- номера соответствующих им фрагментов. Видно, что последовательность ребер $2$, $1$, $3$ является Эйлеровым путем.

Для решения оставшихся подзадач, мы будем находить некоторое множество значений $d$, каждое из которых будем проверять за линейное время алгоритмом из предыдущего абзаца.

\subsection*{Подзадача 4}
Чтобы решить четвертую подзадачу, нужно заметить, что по принципу Дирихле всегда будет существовать хотя бы один фрагмент, содержащий хотя бы две перекладины. Поэтому, если искомый порядок существует, расстояние $d$ между соседними перекладинами должно быть равно расстоянию между соседними перекладинами на этом фрагменте. Воспользуемся алгоритмом и проверим, подходит ли данное $d$.

\subsection*{Подзадача 3}
Чтобы решить третью подзадачу, можно перебрать два варианта:
\begin{itemize}
\item Если фрагмент номер $1$ не будет стоять самым последним, то можно перебрать номер фрагмента $i$, который будет стоять сразу после фрагмента $1$, и проверить $d = r_1 + l_i$.
\item Если фрагмент номер $1$ будет стоять самым последним, то фрагмент номер $2$ не будет стоять последним. Можно перебрать номер фрагмента $i$, который будет стоять после фрагмента $2$, и проверить $d = r_2 + l_i$.
\end{itemize}

Таким образом, мы проверим $O(n)$ вариантов, каждый за время $O(n)$. Итого, решение работает за $O(n^2)$.

\subsection*{Подзадача 5}
Чтобы решить пятую подзадачу, можно проверить все значения $d$ от $0$ до $200$. Дополнительно ускорить решение можно проверяя только значения $d$ от $0$ до $\lfloor\frac{100 \cdot n}{n - 1}\rfloor$.

\subsection*{Подзадача 6}
Наконец, чтобы решить задачу на полный балл, нужно было научиться проверять только $O(1)$ различных вариантов $d$. Рассмотрим пары $(l_{q_i}, r_{q_{i + 1}})$ и отсортируем их в порядке возрастания $l$. Заметим, что так как суммы значений в каждой паре равны, значения $r$ будут убывать. Среди всех значений $l$ в рассмотренных парах не присутствует ровно одно, соответствующее первому фрагменту. Аналогично, среди всех значений $r$ отсутствует значение, соответствующее последнему фрагменту. Таким образом, можно отсортировать все значения $l$ в порядке возрастания, все значения $r$ в порядке убывания и проверить $4$ варианта значения $d$: сумма одного из двух минимальных значений $l$ и одного из двух максимальных значений $r$.

\end{tutorial}
