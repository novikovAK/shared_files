\begin{problem}{Книжная полка}{standard input}{standard output}{1 second}{512 megabytes}

Даниил любит порядок, закономерности и чтение. У него есть $n$ книг, в каждой из которых есть несколько закладок. В каждой книге есть хотя бы одна закладка, ведь Даниил очень любит читать. Чтобы навести порядок, он решил поставить все $n$ книг на одну полку. Книги будут стоять на полке слева направо, обложка к обложке. Если пронумеровать книги слева направо, начиная с $1$, лицевая обложка первой книги будет касаться тыльной обложки второй книги, лицевая обложка второй книги будет касаться тыльной обложки третьей, и так далее. 

Осталось расставить книги так, чтобы расстановка удовлетворяла какой-нибудь закономерности. Даниил знает, сколько в каждой книге страниц, а также знает, на какой позиции в книге находится каждая закладка. Позиция закладки в книге обозначается количеством страниц от начала книги до закладки. Когда Даниил расставит книги на полке, он рассмотрит все закладки слева направо. Сначала будут идти закладки из самой левой книги, затем из следующей, и так далее. Обратите внимание, что закладки из одной книги будут идти в порядке убывания позиции закладки в книге. Затем, Даниил найдет количество страниц между каждой парой соседних закладок. Если соседние закладки лежат в соседних книгах, то количество страниц между ними равно количеству страниц между левой закладкой и лицевой обложкой левой книги плюс количество страниц между правой закладкой и тыльной обложкой правой книги. Даниил хочет, чтобы количество страниц между парой соседних закладок было равно для всех пар соседних закладок.

Помогите Даниилу выбрать порядок книг, в котором они должны стоять на полке слева направо, чтобы желаемое свойство выполнялось.

\InputFile
В первой строке дано одно число $n$~--- количество книг ($1 \le n \le 100\,000$).

Далее дано описание $n$ книг. В первой строке описания книги даны два целых числа $m_i$ и $s_i$~--- количество закладок и количество страниц в $i$-й книге ($1 \le m_i \le 100\,000$, $0 \le s_i \le 10^9$). В следующей строке даны $m_i$ целых чисел $p_{i, j}$~--- позиции закладок, $p_{i, j}$ равно количеству страниц от начала книги до $j$-й закладки ($0 \le p_{i, 1} < p_{i, 2} < \dots < p_{i, m_i} \le s_i$).

Сумма всех $m_i$ не превышает $100\,000$.

\OutputFile
Если расставить книги желаемым образом возможно, в первой строке выведите <<\t{Yes}>>, а во второй строке выведите перестановку чисел от $1$ до $n$~--- номера книг в порядке, в котором они должны стоять на полке слева направо. Если существует несколько подходящих порядков, можете вывести любой.

Иначе, в единственной строке выведите <<\t{No}>>.


\newpage
\Scoring
\begin{tabular}{|c|c|c|c|}
\hline
Подзадача & Баллы & Ограничения & Необходимые подзадачи \\
\hline
1 & 8 & $n \le 8$, $m_i = 1$, $s_i \le 100$ & {---} \\
\hline
2 & 8 & $n \le 8$, $s_i \le 100$ & 1 \\
\hline
3 & 21 & $n \le 1\,000$ & 1, 2 \\
\hline
4 & 21 & $\sum m_i > n$ & {---} \\
\hline
5 & 21 & $s_i \le 100$ & 1, 2 \\
\hline
6 & 21 & Без дополнительных ограничений & 1-5 \\
\hline
\end{tabular}

\Examples

\begin{example}
\exmpfile{example.01}{example.01.a}%
\exmpfile{example.02}{example.02.a}%
\exmpfile{example.03}{example.03.a}%
\exmpfile{example.04}{example.04.a}%
\exmpfile{example.05}{example.05.a}%
\end{example}

\end{problem}

