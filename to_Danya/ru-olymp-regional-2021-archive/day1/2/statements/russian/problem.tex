\begin{problem}{Разбиение таблицы}{стандартный ввод}{стандартный вывод}{1 секунда}{512 мегабайт}

Рассмотрим таблицу из $n$ строк и $m$ столбцов, в клетки которой по строкам записаны числа от~$1$ до~$n \cdot m$. Сначала заполняется первая строка слева направо, затем вторая, и так далее. Другими словами в клетку $(r, c)$ записано число $(r - 1) \cdot m + c$. 

На рисунке приведен пример такой таблицы для $n = 3$, $m = 5$.

\begin{center}
\begin{tabular}{|c|c|c|c|c|}
\hline
1&2&3&4&5\\
\hline
6&7&8&9&10\\
\hline
11&12&13&14&15\\
\hline
\end{tabular}
\end{center}

Требуется разделить таблицу одним вертикальным или горизонтальным разрезом, проходящим по сторонам клеток, так чтобы сумма чисел в получившихся частях таблицы отличалась как можно меньше. В этой задаче в одном тесте вам придётся ответить на несколько запросов об оптимальном разрезании таблицы.

\InputFile
В первой строке ввода задано целое число $t$~--- количеcтво запросов ($1 \le t \le 10^5$). 

В следующих $t$ строках заданы по два числа $n$, $m$ ($1 \le n, m \le 10^9$, $2 \le n \times m \le 10^9$).

\OutputFile
В $t$ строках выведите ответы на запросы, по одному на строке. 

Ответ на каждый запрос должен быть выведен в формате <<\texttt{D $x$}>>, где \t{D}~--- это <<\texttt{V}>>, если нужно резать по вертикали, <<\texttt{H}>>~--- если по горизонтали, а $x$~--- номер столбца или строки, перед которым надо сделать разрез.
Строки пронумерованы от $1$ до $n$, столбцы пронумерованы от $1$ до $m$.

Если правильных ответов несколько, то надо вывести вариант с вертикальным разрезом, если он есть, а если и после этого вариантов несколько, то из вариантов с различными $x$ следует выбрать тот, в котором $x$ меньше.

\Scoring
Баллы за каждую подзадачу начисляются только в случае, если все тесты для этой
подзадачи и необходимых подзадач успешно пройдены.

\begin{center}
\renewcommand{\arraystretch}{1.3}
\begin{tabular}{|c|c|c|c|c|}
\hline
\textbf{Подзадача} & 
\textbf{Баллы} & 
\textbf{Ограничения} & 
\parbox{3cm}{\textbf{\centering\\Необходимые\\подзадачи\\\vspace{2mm}}} & 
\parbox{3cm}{\textbf{\centering\\Информация\\о проверке\\\vspace{2mm}}} 
\\ \hline
1 & 20 & $t = 1$, $1 \le n, m \le 100$ &  & полная \\ \hline
2 & 14 & $t = 1$, $1 \le n, m \le 2\,000$ & 1 & первая ошибка \\ \hline
3 & 15 & $t = 1$, $1 \le n, m \le 10^7$ & 1, 2 & первая ошибка \\ \hline
4 & 16 & $1 \le t \le 1\,000$, $1 \le n \times m \le 10\,000$ & 1 & первая ошибка \\ \hline
5 & 15 & $1 \le t \le 100\,000$, $n = 1$, $1 \le m \le 10^9$ & & первая ошибка \\ \hline
6 & 20 & $1 \le t \le 100\,000$, $1 \le n, m \le 10^9$ & 1--5 & первая ошибка \\ \hline
\end{tabular}

\end{center}

\Example

\begin{example}
\exmpfile{example.01}{example.01.a}%
\end{example}

\end{problem}

