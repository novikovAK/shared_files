\begin{tutorial}{Разбиение таблицы}

\medskip
\textit{Автор задачи: Даниил Орешников}
\medskip

\subsection*{Подзадача 1}
Переберем линию разделения. Затем за $n \cdot m$ посчитаем сумму в разделенной таблице. Выберем лучший разрез. В сумме решение работает за $n \cdot m \cdot (n + m)$.

\subsection*{Подзадача 2}
Заметим, что при перемещении линии разреза на $1$ в каждой из половин добавляется или удаляется только $O(n)$ или $O(m)$ клеток. Значит можно перебрать все варианты за $O(nm)$.

\subsection*{Подзадачи 3 и 4}
Зафиксируем какой-нибудь разрез по вертикале перед столбцом $k$, посчитаем сумму в полученном прямоугольнике, используя сумму арифметической прогрессии:

$$a_1 + a_2 + a_3 + \ldots + a_x = \frac{(a_1 + a_x)}{2} \cdot x$$

Применяя, получаем:
$$1 + 2 + \ldots + k + (m + 1) + (m + 2) + \ldots + (m + k) + \ldots + (m \cdot (n - 1) + 1) + (m \cdot (n - 1) + 2) + \ldots + (m \cdot (n - 1) + k)=$$

$$=\frac{1 + k}{2} \cdot k + \frac{(m + 1) + (m + k))}{2} \cdot k + \ldots + \frac{((m \cdot (n - 1) + 1) + (m \cdot (n - 1) + k)}{2} \cdot k$$


Полученная сумма также является арифметической прогрессией с шагом $m \cdot k$, значит она равна:

$$\frac{\frac{1 + k}{2} \cdot k + \frac{((m \cdot (n - 1) + 1) + (m \cdot (n - 1) + k)}{2} \cdot k}{2} \cdot n$$

Cумму в горизонтальном разрезе найдем по более простой формуле:

$$1 + 2 + \ldots + m + (m + 1) + (m + 2) + \ldots + (m + m) + \ldots + (m \cdot (k - 1) + 1) + (m \cdot (k - 1) + 2) + \ldots + (m \cdot (k - 1) + m)$$

$$\frac{1 + m \cdot k}{2} \cdot mk$$

Таким образом, за $O(1)$ можно посчитать сумму при фиксированном разрезе, а значит, можно найти лучший разрез за $(n + m)$.

\subsection*{Подзадача 5}
Здесь арифметическая програссия одномерная и надо найти такое $k$, чтобы  

$$\frac{1 + k}{2} \cdot k \approx \frac{1 + m}{4} \cdot m$$

\subsubsection*{Решение 1}
Упрощая, получим квадратное уравнение на $k$, и найдем лучшее $k$ за $O(1)$. Из-за округлений возможна погрешность, поэтому нужно перебрать ответ в диапазоне $\pm 1$.
\subsubsection*{Решение 2} 
Можно найти лучшее $k$, используя двоичный поиск (сумма первых $k$ значений возрастает при увеличении $k$).

\subsection*{Подзадача 6}
Совместим идеи в подзадачах 4 и 5:
Посчитаем сумму при фиксированной границе раздела:

$$\frac{\frac{1 + k}{2} \cdot k + \frac{((m \cdot (n - 1) + 1) + (m \cdot (n - 1) + k)}{2} \cdot k}{2} \cdot n \approx \frac{1 + n \cdot m}{4} \cdot nm$$

\subsubsection*{Решение 1 за $O(1)$ на запрос}
Упрощая, получим квадратное уравнение на $k$, и найдем лучшее $k$ с точностью до $\pm 1$, так как знак приблизительно равно. 

\subsubsection*{Решение 2 за $\log_2{max(n, m)}$ на запрос} 
Можно найти лучшее $k$, используя двоичный поиск, так как значение слева возрастает, при увеличении $k$.

\end{tutorial}
