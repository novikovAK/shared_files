\begin{problem}{Изменённая ДНК}{стандартный ввод}{стандартный вывод}{1 секунда}{512 мегабайт}

Биологи обнаружили новый живой организм и решили изучить его ДНК. ДНК кодируется последовательностью символов <<\t{A}>>, <<\t{G}>>, <<\t{C}>> и <<\t{T}>>. 

Так как строка, кодирующая ДНК, часто очень длинная, для её хранения применяют RLE-кодирование. А именно, каждый блок, состоящий из двух или более идущих подряд одинаковых символов, заменяется на число, равное длине этого блока, после которого записывается соответствующий символ. Например, последовательность <<\t{AAAGGTCCA}>> в закодированной форме имеет вид <<\t{3A2GT2CA}>>.

В результате экспериментов, проводимых в лаборатории, ДНК может мутировать. Каждая мутация "--- это либо удаление одного символа из последовательности, либо добавление одного символа, либо замена одного символа на другой.

Уходя вечером из лаборатории, учёный записал ДНК в закодированной форме. Когда он вернулся на работу утром, он обнаружил, что в ДНК произошла ровно одна мутация. Теперь ученых интересует, какая минимальная и максимальная длина может получиться у новой ДНК в закодированной форме.

Требуется по заданной ДНК в закодированной форме определить, какая мутация может привести к тому, что у новой ДНК будет закодированная форма минимальной возможной длины, а какая "--- к тому, что у новой ДНК будет закодированная форма максимальной возможной длины.

\InputFile
В единственной строке входа находится строка $s$, состоящая из цифр и букв 
<<\t{A}>>, <<\t{G}>>, <<\t{C}>> и <<\t{T}>> "--- закодированная ДНК. 

Гарантируется, что это строка является корректной закодированной записью некоторой строки из символов <<\t{A}>>, <<\t{G}>>, <<\t{C}>> и <<\t{T}>>.

\OutputFile
В первой строке выведите мутацию, после которой закодированная строка имеет минимальную длину. Выведите:
\begin{itemize}[noitemsep,leftmargin=2cm]
\item \t{1 $x$ Z},  если надо вставить символ \t{Z} так, чтобы слева от него было ровно $x$ старых символов. Символ \t{Z} должен быть из множества \{\t{A}, \t{C}, \t{G}, \t{T}\}.
\item \t{2 $x$}, если надо удалить символ с номером $x$ из последовательности.
\item \t{3 $x$ Z}, если надо заменить символ с номером $x$ заменить на символ \t{Z}. Символ \t{Z} должен быть из множества \{\t{A}, \t{C}, \t{G}, \t{T}\}. При этом на этом месте до мутации обязательно должен был находиться символ, не равный \t{Z}. 
\end{itemize}

В следующей строке выведите мутацию, после которой закодированная строка имеет максимальную длину, в таком же формате.

Если подходящих ответов несколько, можно вывести любой из них.


\Scoring
Баллы за каждую подзадачу начисляются только в случае, если все тесты для этой
подзадачи и всех необходимых подзадач успешно пройдены. 

Обозначим за $n$ длину закодированной строки, а за $L$ длину исходной строки.

\begin{center}
\renewcommand{\arraystretch}{1.3}
\begin{tabular}{|c|c|c|c|c|}
\hline
\textbf{Подзадача} & 
\textbf{Баллы} & 
\textbf{Ограничения} & 
\parbox{3cm}{\textbf{\centering\\Необходимые\\подзадачи\\\vspace{2mm}}} & 
\parbox{3cm}{\textbf{\centering\\Информация\\о проверке\\\vspace{2mm}}}\\
\hline
1 & 9 & $1 \le n \le L \le 10$ & & полная \\ \hline
2 & 17 & $1 \le n \le 100$, $1 \le L \le {10}^4$ & 1 & первая ошибка \\ \hline
3 & 21 & $1 \le n \le 1000$, $1 \le L \le {10}^5$ & 1, 2 & первая ошибка \\ \hline
4 & 11 & $1 \le n \le {10}^5$, $1 \le L \le {10}^7$ & 1--3 & первая ошибка \\ \hline
5 & 42 & $1 \le n \le {10}^5$, $1 \le L \le {10}^9$ & 1--4 & первая ошибка \\ \hline
\end{tabular}
\end{center}


\Example

\begin{example}
\exmpfile{example.01}{example.01.a}%
\end{example}

\Explanation
Исходная последовательность имела вид <<\t{AAAAACAAAAACC}>>. 

Первая операция превращает её в последовательность <<\t{AAAAAAAAAAACC}>>, которая кодируется как <<\t{11A2C}>>. Эта закодированная последовательность имеет минимальную возможную для этого теста длину, равную $5$. 

Вторая операция превращает её в последовательность <<\t{AACAAACAAAAACC}>>, которая кодируется как <<\t{2AC3AC5A2C}>>. Эта закодированная последовательность имеет максимальную возможную для этого теста длину, равную $10$.

\end{problem}

