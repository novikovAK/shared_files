\begin{tutorial}{Изменённая ДНК}

\medskip
\textit{Автор задачи: Олег Христенко}
\medskip

\subsection*{Общая идея}
Для того, чтобы получить минимальную длину, всегда выгодно удалить какой-то элемент, а чтобы получить максимальную длину, нужно добавить какой-то элемент. Докажем это. 

Для минимума. Пусть мы заменили какой-то символ. Вместо этого удалим этот символ. Длина каждого из соседних блоков не изменится, либо они сольются в один. В любом случае длина закодированной строки получится меньше или равна, чем при замене. Аналогично, вместо добавления символа всегда выгоднее удалить символ перед ним.

Для максимума. Пусть мы заменили какой-то символ. Тогда вместо этого добавим в эту позицию тот символ, на который мы заменили. Длина каждого из соседних блоков не уменьшится, значит длина закодированной версии полученной строки будет не меньше. Аналогично вместо удаления символа всегда выгоднее добавить символ перед ним.

Перейдем теперь к рассмотрению того, как решить, какую именно операцию сделать.

\subsection*{Подзадачи 1 и 2}
Для того, чтобы решить первые две подзадачи, достаточно перебрать место, в которое нужно вставить новый элемент, или удалить существующий. После фиксирования изменения, можно посчитать длину закодированной строки полным проходом по ней. Асимптотика этого решения $O(L^2)$.

\subsection*{Подзадача 3}
Для того, чтобы решить третью подзадачу, нужно заметить, что каждый блок одинаковых символов, изменился не сильно. Поэтому после изменения символа достаточно для каждого исходного блока посчитать новую длину этого блока или новых блоков, которые появились после изменения. Асимптотика решения $O(n \cdot L)$.

\subsection*{Подзадача 4}
Для того, чтобы решить четвертую подзадачу, нужно заметить, что после изменения меняется только блок, в котором произошло изменение, или соседние с ним блоки. Для всех остальных блоков длина не меняется, и её можно не пересчитывать. Асимптотика решения $O(L)$.

\subsection*{Подзадача 5}
Для того, чтобы решить пятую подзадачу, нужно научиться понимать, как правильно удалять и добавлять символы. 

Если в строке есть блок длины $1$, соседние блоки которого содержат одинаковые символы, то при удалении этого символа эти блоки объединятся в один, и длина закодированной строки уменьшится. Если в строке есть блок длины $1$, то при его удалении длина строки уменьшится на $1$. Если в строке есть блок длины $2$, то при удалении символа из него, в его кодировке больше не нужно будет писать число, и длина строки уменьшится на $1$. Если в строке есть блок длины ${10}^x$, то при удалении символа из него длина числа уменьшится на $1$. Это все возможные способы уменьшения длины при удалении символа. 

При добавлении символа можно поставить его в начало для создания нового блока длины $1$, чтобы длина строки увеличилась на $1$. Также можно какой-то блок разделить на два, если добавить другой символ внутри блока. Для блока длины до $20$, достаточно перебрать, где именно будет новый символ. Если длина блока от $2 \cdot {10}^x$ до ${10}^{x + 1}$, то из него можно выделить блок размера ${10}^x$, и увеличить длину строки на $x + 3$. Если длина блока от ${10}^x + {10}^{x - 1}$ до $2 \cdot {10}^x$, то из него можно выделить блок длины ${10}^x$, и увеличить длину строки на $x + 2$. Если длина блока от ${10}^x$ до ${10}^x + {10}^{x - 1}$, то из него можно выделить блок длины $8 \cdot {10}^{x - 1}$, и увеличить длину строки на $x + 1$. Это все возможные способы увеличить длину строки при добавлении символа.

Для каждого блока проверим все оптимальные способы добавления и удаления символа и выберем оптимальный способ среди всех. Так как для каждого блока оптимальных способов всего $O(1)$, то асимптотика решения $O(n)$.


\end{tutorial}
